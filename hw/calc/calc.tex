\documentclass[12pt]{article}

\title{Object Calculator}
\author{
	Evan M Drake \\
	\texttt{drakeev@butte.edu}
}


\usepackage{snow-preamble}

%%%%%%%%%%%%%%%%%%%%%%%%%%%%%%%%%%%%
\begin{document}
	\maketitle

	%% body start

	\section{Reverse Polish Notation}
		\indent Reverse polish notation is a prefix notation style of computation.
		Of the form $+ 5 2$ is the equivalent expression to $5 + 2$.
		RPN is formally based on the stack data structure and utilizing its efficiency to add speed to calculation.
		As the first thing you pop is the operation which will infer the quantity and type of the following data.
		This time you will be required to have the calculator located in an object.
		Something like the below:

		\begin{lstlisting}[style=code, language=C++]
class Calc
{
private:
	stack<int> st;
	void addition(int, int);
	void subtraction(int, int);
public:
	int evaluate_stack(stack<int>);
	Calc();
};
		\end{lstlisting}

	\section{Your Problem}
		\indent Use what you know to build an RPN calculator in C++.
		This calculator will not take user input while running,
			rather you will fill a stack manually in the program prior to run time.
		You will build this calculator to infer quantity of inputs and type from the operation given on the first pop.
		You will only use Natural numbers for now.
		You will be evaluated (graded) only on how well you infer the quality and type of data coming from the stack.
		Meaning you should focus on the logic of how a stack works and how that might help you make assumptions.

	\section{Turn ins}
		It goes without saying but you must turn in both code and an English portion.

	%% body finish

	\nocite{*}

	\newpage % bib
	\backmatter
	\addcontentsline{toc}{section}{References}
	\printbibliography

	\newpage % index
	\printindex

	\newpage % appendices
	\appendix

\end{document}

