\documentclass[12pt]{article}

\title{Syllabus}
\author{
	Evan M Drake \\
	\texttt{drakeev@butte.edu}
}


\usepackage{snow-preamble}

%%%%%%%%%%%%%%%%%%%%%%%%%%%%%%%%%%%%
\begin{document}
	\maketitle
	\tableofcontents

	\newpage

	%% body start
	\section{Course Description}
		\indent This course is an introduction to the discipline of computer science,
			with a focus on the design and implementation of algorithms to solve simple problems using a high-level programming language.
		Topics include fundamental programming constructs,
			problem-solving strategies, debugging techniques,
			declaration models, and an overview of procedural and object-oriented programming languages.
		Students will learn to design, implement, test, and debug algorithms using pseudocode and a high-level programming language.
	\section{Student Learning Outcomes}
		\begin{enumerate}
			\item Design, implement, test, and debug computer programs using basic computation, simple Input/Output (I/O), standard conditional and iterative structures, and functions.
			\item Use pseudocode and a high-level programming language to implement, test, and debug algorithms that solve simple problems.
			\item Summarize the evolution of programming languages and describe how this evolution has led to the programming paradigms in use today.
			\item Identify and demonstrate different forms of variable binding, visibility, scoping, and lifetime management.
		\end{enumerate}
	\section{Course Materials}
		\indent We will be using a mix of free online resources in lieu of a textbook, including:
		\begin{enumerate}
			\item \url{https://cppreference.com/}
			\item \url{https://www.w3schools.com/cpp/}
			\item \url{https://www.youtube.com}
		\end{enumerate}
	\section{Course Requirements}
	\subsection{Homework}
		\indent Readings, Quizzes and other Assessments will reinforce class concepts

	\subsection{Lab Work}
		\indent Will be assigned as either individual or group work

	\subsection{Coding Assignments}
		\indent Students will complete assignments during the semester that will span multiple weeks.

	\subsection{Exams}
		\indent There will be possibly exams during the semester and one (1) final project.
		There will be at least two papers that students will complete.

	\section{Academic Resources}
		The Center for Academic Success (CAS) offers free academic help to all enrolled Butte College students. \url{www.butte.edu/cas}

		 Locations \& Hours (also found in EDUC 310 Canvas course)\:

		Chico Center – CHC 230. M\/T: 9am to 6:00pm, W\/Th: 10am to 8 pm, F: 9am to 12pm

		Glenn County Center – GCC 113. M\/W: 10:30am to 2:30pm, T\/Th: 11am to 7pm

		Main Campus – LRC 203. M-Th: 8 am to 5 pm, F: 8 am to 2 pm

		DROP-IN TUTORING – No appointment needed
		DROP-IN COMPUTER SUPPORT – CHC 231 and GCC 113
		LAPTOP CHECK OUT AT ANY LOCATION
		CRITICAL SKILLS WORKSHOPS – Schedules available at all CAS locations, from \url{www.butte.edu/cas/workshops}, and in Canvas (EDUC 310)
		CRITICAL SKILLS FOR COLLEGE SUCCESS ($\frac{1}{2}$-unit course) – EDUC 10 or 110
		SUCCESS COACHING with CAS Faculty 
		GROUP STUDY ROOMS – Main campus and Glenn County Center 
		CAS TIP SHEETS – \url{www.butte.edu/cas/tipsheets/}

		To access all services, students register for EDUC 310, which is free, non-graded, and no-credit. Follow this link to enroll in control \#3286.

		Butte College Library services never close with our 24/7 Live Chat feature, plus you can easily search for library resources from your home computer.
		Our librarians at the Main Campus and Chico Center libraries can help you with locating research guides, books, journal articles, newspaper articles, and much more.
		The Library also has laptop computers and mobile WiFi hotspots available for week-long check-out!


	%% body finish

	\nocite{*}

	\newpage % bib
	\backmatter
	\addcontentsline{toc}{section}{References}
	\printbibliography

	\newpage % index
	\printindex

	\newpage % appendices
	\appendix

\end{document}

